\documentclass{article}
\usepackage{amsthm,amsfonts,mathtools, amssymb}
\usepackage{enumerate}
\usepackage{outlines}
\usepackage{changepage}
\textwidth=6.25in
\textheight=9.5in
\headheight=0in
\headsep=.5in
\hoffset  -1in
\topmargin -.75in

\nofiles

\setlength{\parskip}{0.0cm}
\setlength{\parindent}{0.0cm}   % Don't indent the paragraphs

% \begin{adjustwidth}{2em}{0pt}
% \end{adjustwidth}

\begin{document}
% \begin{flushright}
%     Matt Gautreau\\
%     Math 443\\
%     Due:\\
% \end{flushright}

\begin{outline}
    \1 Introduction
    \1 Group theory basics
        \2 Group Definition
        \2 Group Actions
            \3 Orbit
            \3 Stabilizer
    \1 fundamental concepts of computational group theory
        \2 Algorithms need to be described imperitavly, but must be mathematically valid.
        \2 generating set
        \2 orbit calculations
        \2 Schreier trees
        \2 Stabilizer Calculation
        \2 Schreir-Sims \& Basis and Strong Generating Sets
        \2 differences for matrices
    \1 Haskell algorithm implementations
        \2 Explanation of Data Structures used
        \2 Group and GroupAction typeclasses
            \3 instances for permutations, matrices
        \2 Performance Analysis
            \3 time tables
            \3 Test cases?
                \4 Difficult / computational enough
                \4 symmetric groups are usually hardest - GAP usually specializes for these
        \2 Code snipets(?)
            \3 type signatures
            \3 typeclass \& type constrains
                \4 laws not enforacble by haskell
            \3 important functions (fold ruduction functions in schreir trees, etc)
        \2 Difficulties
    \1 Other implementations
        \2 simply need to mention they exist, if that
        \2 GAP
            \3 unless used for verification, might be able to be left out
        \2 HaskellForMaths
            \3 comment on any inspiration, use, etc
            \3 overlapping functionality, differences in implementation between the overlap
\end{outline}



\end{document}
